% В ColorSet.Hrc сделано изменение
% <Files Scheme="Tex" Exts="/\.(tex)|(cls)|(sty)$/i" Color="dText">
%
% А теперь о самой раскраске:
% 1) поскольку ТеХ по-особому относится к скобкам {} и [], то их содержимое
%    выделяется красненьким, чтобы сразу была видна парность/непарность
%    этих скобок.
% 2) Все формулы пишутся на черном фоне
% 3) Символы, которых быть не должно - на красном фоне
% 4) Цифры - зеленые
% 5) Основной текст - дефолтного цвета.
% 6) Спецсимволы - белые
% 7) Команды - желтые
% 8) Примечания - непонятного серо-зеленого цвета.
%
% Чтобы долго не извращаться, я просто кусков из реальной статьи нарезал


\documentstyle[12pt]{article}
\begin{document}
\title{Excitation of plasma oscillations by moving Josephson vortices in
layered superconductors}
\author{ }
\maketitle

The density of the superconducting current between layers $n$ and $n+1$ is
given by $$j^{(s)}_{\perp n}(\varphi_n)=j_c \sin{\varphi_n}$$,
where $\varphi_n$ is the gauge invariant phase difference. We shall adopt an
exactly solvable model substituting $\sin{\varphi_n}$ by the saw-tooth
function
\begin{equation}
%
% синий цвет фона {equation} - для симметрии с \end{equation},
% хотя, это, конечно, кривизна.
%
j^{(s)}_{\perp n}(\varphi_n)=j_c\arcsin{\sin{\varphi_n}},
\end{equation}

%
% а вот и эти самые злополучные \begin{equation}
% и \end{equation}
%
In the displacement current we take into account only the component along
$z$ axis, since due to the strong anisotropy of a layered superconductor the
plasma frequency in the direction parallel to the layers, $\Omega_p =c/
\lambda$, is much larger, than typical frequencies of the problem, which
are of the order of the plasma frequency for perpendicular direction,
$\omega_p= c_z/\lambda_c$. Here $c_z =c/\sqrt{\epsilon}$, $\epsilon$ is a
dielectric constant in $z$ direction, $\lambda$ and $\lambda_c =c/
\sqrt{8\pi d j_c}$ are the screening lengths for the current flowing
parallel and perpendicular to the layers, respectively. Then we get the
equations
\begin{eqnarray}
j^{(s)}_{\perp n}(\varphi_n)-
\frac{c^2}{8\pi d}
%
% а вот и \frac{}{}
% хочется, чтобы в \frac{c^2}+... буквы "\frac" были бы на красном фоне,
% поскольку нету второй пары скобок {}
%
\frac{\partial^2\varphi_n}{\partial x^2} +
2\lambda_c^2\frac{\partial}{\partial x} (P_{n+1} - P_n) =
-\frac{\lambda_c^2}{c^2}
\frac{\partial}{\partial t}
\left(
  \frac{            % такие многострочные фраки должны правильно
    4\pi            % отрабатываться! Как это сделать с одновременной
    \sigma_{\perp}  % закраской фона в красный цвет у неправильных фраков
   }                % я просто не знаю :(
   {
     \epsilon
   }+
  \frac{\partial}{\partial t}
\right)
\varphi_n
\\ % разделитель
\frac{c^2}{2d^2}
\frac{\partial}{\partial x}
(\varphi_n -\varphi_{n-1}) +\Omega_p^2 P_n -
  \frac{c^2}{d^2}(P_{n+1}+P_{n-1}-2P_n
)=
-\frac{\partial}{\partial t}
\left(4\pi \sigma_{\|} +
  \frac{\partial}{\partial t}
\right) P_n.
\end{eqnarray}
%
% eqnarray = equation array
% В принципе, eqnarray может быть вложен в equation.
%
The solution of these equations yields spatial and temporal dependencies of
electric and magnetic fields.

\end{document}
